\documentclass[12pt]{article}
\usepackage{odia}
\usepackage{pifont,mdframed}
\usepackage{hyperref}

\usepackage{geometry}
 \geometry{
 a4paper,
 total={170mm,257mm},
 left=20mm,
 top=20mm,
 }



\newcommand*{\plogo}{} % Generic publisher logo

%----------------------------------------------------------------------------------------
%	TITLE PAGE
%----------------------------------------------------------------------------------------

\newcommand*{\titleGM}{\begingroup % Create the command for including the title page in the document
\hbox{ % Horizontal box
\hspace*{0.2\textwidth} % Whitespace to the left of the title page
\rule{1pt}{\textheight} % Vertical line
\hspace*{0.05\textwidth} % Whitespace between the vertical line and title page text
\parbox[b]{0.75\textwidth}{ % Paragraph box which restricts text to less than the width of the page

{\noindent\Huge\bfseries The Odia-\TeX\ Manual}\\[2\baselineskip] % Title
{\large \textit{Typesetting in Odia using \TeX{}, \LaTeX}}\\[4\baselineskip] % Tagline or further description
{\Large \textsc{SOUMYASHANT NAYAK}} % Author name

\vspace{0.5\textheight} % Whitespace between the title block and the publisher
{\noindent \plogo}\\[\baselineskip] % Publisher and logo
}}
\endgroup}

%\\[0.5\baselineskip]

\def\odtex{Odia-\TeX}
\def\obrace{\char"7B}
\def\cbrace{\char"7D}

%----------------------------
% Load fonts
%----------------------------
\font\hugeknrk=od36 at 72pt
\font\largeknrk=od24 at 24pt
\font\bigknrk=od17 at 17.28pt
\font\knrk=od12 at 12pt

\font\hugecttck=odss36 at 72pt
\font\largecttck=odss24 at 24pt
\font\bigcttck=odss17	at 17.28pt
\font\cttck=odss12 at 12pt

%-----------------------------
% Warning Environment
%-----------------------------

\newenvironment{warning}
  {\par\begin{mdframed}[linewidth=2pt,linecolor=red]%
    \begin{list}{}{\leftmargin=1cm
                   \labelwidth=\leftmargin}\item[\Large\ding{43}]}
  {\end{list}\end{mdframed}\par}




\begin{document}
\titleGM
\thispagestyle{empty}
\clearpage
\tableofcontents
\thispagestyle{empty}
\clearpage

\pagenumbering{knrk}

\section*{Summary}

This article gives an overview of the \odtex-package which can be used with \TeX{}, \LaTeX{} 
for typesetting text in Odia {\knrk{}(ow[\odnukta aA)}. It also lays down an intuitive scheme for transliterating words
from Odia into English and describes usage of the Python script {\tt od2tex} to convert the transliterated text to 
Odia-TeX syntax. Additionally, one may use \odtex\ in combination with other packages, to typeset mixed language documents. 
\odtex{} is a fork of the Oriya-\TeX\ project by Jeroen Hellingman and this manual borrows extensively from the original
article written by him.

\section*{License}

The Odia-\TeX\ fonts are Copyright \copyright\ 1999-2016 Jeroen Hellingman, 
Soumyashant Nayak. The pre-processor {\tt od2tex} is Copyright \copyright\ 2016 Soumyashant Nayak. However,
 they may be freely distributed in the spirit 
of the GNU General Public License. The only restriction is that you cannot place further restrictions on this or 
derived works and you should give others you give this software or derived works to, everything, 
including the sources. Of course, works you print using this font are not derived 
works, but if you add or touch up a few characters, or create a post script font, 
based on this design, it is.
\vskip 0.1in
This is still a work in progress and is part of a bigger vision to make user-friendly professional typesetting options
available in Odia. If you would like to contribute to this project, feel free to write to the e-mail address
given below.

\vskip 0.2in

\noindent Soumyashant Nayak\\
\noindent E-mail: {\tt<odia.bhashakosha@gmail.com>}\\
\noindent Github page : {\tt<https://github.com/nsoum/odia-tex>}

\newpage

\section{Introduction}
The Odia alphabet follows the alphabetic order of Sanskrit. This
order is based on phonetic principles. The vowels come first, in pairs of 
a short and long sound, those articulated in the
back of the mouth first. After the `pure' vowels come the `diphtongs'
(in Sanskrit, {\it e} and
{\it o} are supposed to have been diphtongs originally).

\def\OB#1#2{$\hbox{\bigknrk#1}\atop\hbox{\strut\it#2}$}

$$
\vbox{\halign{\it\hfill#\hfill\quad&&\it\hfill#\hfill\quad\cr
\OB{a}{a}	&\OB{aA}{\=a}	&\OB{i}{i}	&\OB{I}{\=\i}	&\OB{u}{u}	&\OB{U}{\=u}	\cr\noalign{\smallskip}
\OB{\odvowelri}{\d r}&\OB{\odvowelrii}{\d{\=r}}&\OB{\odvowelli}{\d l}&\OB{\odvowellii}{\d{\=l}}	\cr\noalign{\smallskip}
\OB{e}{e}	&\OB{E}{ai}	&\OB{o}{o}	&\OB{O}{au}	\cr
}}
$$

\noindent Three vowel modifiers traditionally follow the vowels. These are
{\knrk a/} {\it a\.m}, {\knrk aM} {\it a\d{m}}, and {\knrk aH} {\it a\d{h}},
which originally stood for a nasalisation, a following
nasal, and a following aspiration respectively. 
In Odia, {\knrk H} {\it \d h} is now used to indicate that the following consonant
is doubled.
Here written on the letter {\knrk a} {\it a,}
they can appear on any letter. The modifiers are 
followed by the consonants arranged in rows (called {\it varg}) of related sounds. The rows are ordered
again by point of articulation, with semivowels and sibilants comming at the end, 
and in each row the voiceless sound first, followed by the aspirated, voiced,
voiced and aspirated sound and the corresponding nasal.

$$
\vbox{\halign{\it\hfill#\hfill\quad&&\it\hfill#\hfill\quad\cr
\OB{k}{ka}	&\OB{K}{kha}	&\OB{g}{ga}	&\OB{G}{gha}	&\OB{f}{\.na}	\cr\noalign{\smallskip}
\OB{c}{ca}	&\OB{C}{cha}	&\OB{j}{ja}	&\OB{J}{jha}	&\OB{F}{\~na}	\cr\noalign{\smallskip}
\OB{q}{\d ta}	&\OB{Q}{\d tha}	&\OB{w}{\d da}	&\OB{W}{\d dha}	&\OB{N}{\d na}	\cr\noalign{\smallskip}
\OB{t}{ta}	&\OB{T}{tha}	&\OB{d}{da}	&\OB{D}{dha}	&\OB{n}{na}	\cr\noalign{\smallskip}
\OB{p}{pa}	&\OB{P}{pha}	&\OB{b}{ba}	&\OB{B}{bha}	&\OB{m}{ma}	\cr\noalign{\smallskip}
\OB{Y}{ya}	&\OB{y}{\.ya}	&\OB{r}{ra}	&\OB{l}{la}	&\OB{v}{va}	\cr\noalign{\smallskip}
\OB{z}{\'sa}	&\OB{S}{\d sa}	&\OB{s}{sa}					\cr\noalign{\smallskip}
\OB{h}{ha}	&\OB{L}{\d la}	&	&\OB{w\odnukta}{\d{r}a}	&\OB{W\odnukta}{\d{r}ha}\cr
}}
$$

\noindent as shown in this table, all consonants are thought to be followed by the
short letter {\knrk a} {\it a.}\footnote{The vowels {\knrk\odvowelrii, \odvowelli,}
and {\knrk\odvowellii} are not used in Odia at all. They are borrowed from Sanskrit
(and even in Sanskrit the last one is only introduced for symmetry), and are
included in the alphabet for completeness. The letter {\knrk v} is a
recent invention to distinguish the sound {\it va} from {\it ba} in loanwords.
It is a combination of the vowel {\knrk o} with a secondary {\knrk b}.}

When a vowel follows a consonant, this is indicated by writing a special vowel sign,
attached to the consonant. These vowel signs sometimes even stand before the
consonant they apply to. Here the vowel signs are shown attached to the letter 
{\knrk g} {\it ga\/}:

$$
\vbox{\halign{\it\hfill#\hfill\quad&&\it\hfill#\hfill\quad\cr
\OB{g}{ga}	&\OB{gA}{g\=a}	&\OB{g[}{gi}	&\OB{gX}{g\=\i} &\OB{g]}{gu}	&\OB{gZ}{g\=u}	\cr\noalign{\smallskip}
\OB{g\odsignri}{g\d r}	&\OB{<g}{ge}	&\OB{<g>}{gai}	&\OB{<gA}{go} &\OB{<g*}{gau}	\cr\noalign{\smallskip}
}}
$$

Vowel signs often combine with the consonant they apply to:
The vowel sign for {\it\=a,} {\knrk\dotcircle A}, is often joined to the letter, as
in {\knrk kA} {\it k\=a.}
The vowel sign for {\it i,} {\knrk\dotcircle[}, `sinks' into the outer circle,
for example in 
{\knrk k[} {\it ki}, and the vowel sign for {\it u,} {\knrk\dotcircle]}, can be written
in one continuous stroke, as in {\knrk k]} {\it ku}. Sometimes
the vowel sign for {\it u,} {\knrk\dotcircle]}, becomes {\knrk\dotcircle\odsignuvar},
for example in {\knrk l]} {\it lu}.
On the letters  
{\knrk T} {\it tha\/} and {\knrk D} {\it dha}, the vowel sign for {\it i,}
{\knrk\dotcircle[}, becomes a little hook {\knrk\dotcircle\odsignivar}, written
below the letter: {\knrk T[} {\it thi}, {\knrk D[} {\it dhi}. However, recent
simplifications
of the script have done away with all these irregular vowel signs.

When no vowel follows a consonant at all, not even the short {\knrk a} {\it a,}
this can be indicated with
a sign called virama, {\knrk\dotcircle\odhalant}, for example, {\knrk k\odhalant} {\it k.}
However, this symbol is not
used that often. It is normally omitted at the end of words.

When two or more consonants follow each other, they are combined into a
consonant cluster or conjunct.
Often these conjuncts consist of a smaller version of the second letter subscribed
to the first letter, for example, {\knrk\odshca} {\it \'sca\/}
is used for {\knrk z\odhalant c}. Sometimes,
the outer circle of the subscribed consonant is omitted, as in {\knrk \odLka} {\it \d{l}ka\/}
for {\knrk L\odhalant k}. In many cases, however, the original consonants can hardly
be recognised in the conjunct, for example in {\knrk\odkSa} {\it k\d sa\/}
for {\knrk k\odhalant S}. A fairly complete list of conjuncts is given in Section 8.

The letters {\knrk Y} {\it ya} and {\knrk r} {\it ra\/} are treated in a special way. 
When {\knrk r} {\it ra\/} is the first consonant of a cluster, it appears as the symbol
reph {\knrk\dotcircle\odreph} at
the end of the cluster, for example, {\knrk\odNNa\odreph} {\it r\d n\d na}. When it
is the last in a cluster, it appears as a subscribed hook, {\knrk\dotcircle\odsecra}, for
example in {\knrk p\odsecra} {\it pra.} When the letter {\knrk Y} {\it y\/} is the
last in a cluster,
it appears as the symbol {\knrk\dotcircle\odsecya}, for example, {\knrk k\odsecya} {\it
kya.} This same symbol {\knrk\dotcircle\odsecya} appears after the vowels {\knrk e} 
{\it e\/} and {\knrk o} {\it o\/} to modify their sounds.

The rules for combining consonants into conjuncts used to make typing and typesetting
Odia a very complicated business.
Traditional Odia typefaces in lead require several hundereds of different types.
A computer, however, can easily apply all the rules, and
make it possible to type only the base letters of Odia in their phonetic order.

The spelling and the use of conjuncts is hardly standardized, and it is no exception
to come across spelling variations even inside a single book.

Punctuation in Odia script is similar to that in English, except that
the period is replaced by a standing bar {\knrk\oddanda}, called
{\it danda}, which is separated from the last word by a space to avoid
confusion with the vowel sign for {\knrk aA} {\it \=a}.

Finally, Odia uses its own set of figures for numerals, {\knrk 1 2 3 4 5 6 7 8 9 0}, although
international figures are sometimes used as well, and {\knrk \odpercent} for the percent sign. The symbol for Indian 
rupee, given by {\knrk \odrupee}, is also included in \odtex.


\newpage

\section{\odtex{} fonts}

\def\odalphabet{\\a aA i I u U \odvowelri\ e E o O a/ aM aH\\
k K g G f c C j J F q Q w W N t T d D n\\
p P b B m y Y r l L z S s h\\}
\def\odmatras{k kA k[ kX k] kZ k\odsignri\ <k <k> <kA <k* k/ kM kH}
\def\odfigures{1 2 3 4 5 6 7 8 9 0 }
\def\odsigns{\oddecimaldot{} \odpercent{} \odrupee{} }

In this section, we display the two variants of fonts in the \odtex-package. The alphabets, numerals and some extra signs (decimal dot,
percent sign and the Indian rupee symbol) are included.

\subsection{Konark font}
Konark regular 72 pt\\
{\hugeknrk{}a <kANAk\odreph}\\ \\
Konark regular 24pt\\
{\largeknrk{}a aA i I u U \odvowelri\ \odvowelrii\ \odvowelli\ \odvowellii\ e E o O a/ aM aH}\\ \\
{\largeknrk{}k K g G f c C j J F q Q w W N t T d D n}\\ \\
{\largeknrk{}p P b B m y Y r l L z S s h \odkSa{} w\odnukta{} W\odnukta}\\ \\
{\bigknrk{}\odmatras}\\ \\
{\bigknrk{}\odfigures}\\ \\
{\bigknrk{}\odsigns}

\subsection{Cuttack font}
\font\ssbfXII=cmss10 at 12pt
Cuttack regular 72pt\\ \\
{\hugecttck{}a kqk}\\ \\
Cuttack regular 24pt\\
{\largecttck{}a aA i I u U \odvowelri\ \odvowelrii\ \odvowelli\ \odvowellii\ e E o O a/ aM aH}\\ \\
{\largecttck{}k K g G f c C j J F q Q w W N t T d D n}\\ \\
{\largecttck{}p P b B m y Y r l L z S s h \odkSa{} w\odnukta{} W\odnukta}\\ \\
{\bigcttck{}\odmatras}\\ \\
{\bigcttck{}\odfigures}\\ \\
{\bigcttck{}\odsigns} 


\section{Setting up \odtex\ }
In order to setup \odtex\ on a GNU/Linux system, please follow the instructions below.
\begin{itemize}
\item[(i)] Download and extract the \odtex-package from the github page at \\
{\tt \url{https://github.com/nsoum/odia-tex}}. 
\item[(ii)] Place the folders {\tt odia-fonts-core} and {\tt odia-build-metafont-core} in {\tt texmf/fonts/source}. If you are not using a
local {\tt texmf} directory, you must also set the appropriate permissions.
\item[(iii)] Place the folder {\tt odia-tex-core} in {\tt texmf/tex}, and the folder {\tt odia-latex-core} in \\
{\tt texmf/tex/latex}.
\item[(iv)] In order to generate the fonts, run {\tt tex odmacs.tex} at the commandline. This may take a few minutes depending on the specifications of your computer.
\item[(v)] From the {\tt src} folder in the {\tt odia-tex} directory of Step (i), extract the tarball {\tt od2tex-0.5.tar.gz} containing the {\tt od2tex} source code. Go to the extracted folder and run {\tt sudo python setup.py install} at the commandline to install {\tt od2tex} on your system.
\end{itemize}
Now the Konark and Cuttack fonts are ready to be used with plain \TeX\ and \LaTeX\ . Also the pre-processor {\tt od2tex} is installed and the usage tips may be found by running {\tt od2tex --help} at the commandline.

\section{How to Use \odtex{}}
A set of Odia fonts alone doesn't make up an easy-to-use Odia typesetting system.
One also needs an easy way of typing Odia text. The complex
rules of Odia script makes it cumbersome and error prone to type all
required shapes directly. A computer can handle the composition and allow
the user to type only the base characters in phonetic order. As the rules of
Odia composition are not built into \TeX, the pre-processor {\tt od2tex} will be used to accomplish
this by converting Odia text typed in an ASCII-based transliteration scheme to \odtex\ syntax. The transliteration scheme is
prescribed in \ref{subsection: transliterate}. The details of usage of the preprocessor {\tt od2tex} is described in \ref{subsection: od2tex}.

Before you can start, you will have to load some macros.
Users of plain \TeX\ have to input the file {\tt odmacs.tex} somewhere at the
start of their document. Users of \LaTeX\ can simply use the package {\tt odia} to load the
required macros and fonts. In 4.3, an example is discussed to illustrate various aspects of 
the \odtex-package.




\subsection{Transliteration scheme}
\label{subsection: transliterate}
The following table indicates how to type Odia. The first column indicates the
transliteration typed, the second the Odia letter output, and the third the
scientific transcription. It should be noted that the scientific transcription
is based on the model of Sanskrit, and does not give an exact indication of the
actual pronunciation in Odia.
\vskip 5mm

\bgroup
\def\knrkstrut{\vtop to4pt{}\vbox to10pt{}}
\def\q{\quad\hfill}
\def\x{\vrule\ }
\def\xx{\vrule width.8pt}
\offinterlineskip
\halign{\knrkstrut\xx\ \tt#\q&\x#\q&\x#\q\xx\ &\tt#\q&\x#\q&\x#\q\xx\ &\tt#\q&\x#\q&\x#\q\xx\cr
\noalign{\hrule height.8pt}
a       & {\knrk a} or implicit       	& {\it a}         	&
ka      & {\knrk k}                  		& {\it ka}         	&
pa      & {\knrk p}                   	& {\it pa}        	\cr
A      & {\knrk aA} or {\knrk\dotcircle A}	& {\it \=a}     	&
kha     & {\knrk K}                 		& {\it kha}	        &
pha, fa & {\knrk P}                   	& {\it pha, fa}     \cr
i       & {\knrk i} or {\knrk\dotcircle[}	& {\it i}       	&
ga      & {\knrk g}                   	& {\it ga}          &
ba  & {\knrk b}                  	 	& {\it ba}      \cr
I      & {\knrk I} or {\knrk\dotcircle X}	& {\it \=\i}    	&
gha     & {\knrk G}                   	& {\it gha}         &
bha     & {\knrk B}                   	& {\it bha}         \cr
u       & {\knrk u} or {\knrk\dotcircle]}  	& {\it u}       	&
nga.    & {\knrk f}                   	& {\it \.na}        &
ma      & {\knrk m}                   	& {\it ma}       	\cr
U      & {\knrk U} or {\knrk\dotcircle Z}	& {\it \=u}         &
cha      & {\knrk c}                  		& {\it ca}	        &
ya      & {\knrk Y}                   	& {\it ya}          \cr
Ru       & {\knrk\odvowelri} or {\knrk\dotcircle\odsignri}	& {\it \d{r}}	&
chha     & {\knrk C}                 		& {\it cha}         &
Ja     & {\knrk y}                  		& {\it \.ya}        \cr
RU      & {\knrk\odvowelrii} or {\knrk\dotcircle\odsignrii}     		& {\it \d{\=r}}     &
ja      & {\knrk j}                       & {\it ja}          &
ra      & {\knrk r}                		& {\it ra}   		\cr
L.u       & {\knrk\odvowelli} or {\knrk\dotcircle\odsignli}               & {\it \d{l}}       &
jha     & {\knrk J}                 		& {\it jha}		    &
la 		& {\knrk l}            			& {\it la}          \cr
L.U      & {\knrk\odvowellii} or {\knrk\dotcircle\odsignlii}              & {\it \d{\=l}}     &
nya.  & {\knrk F}                 		& {\it \~na}        &
va     & {\knrk v}	                  	& {\it va}          \cr
e       & {\knrk e} or {\knrk<\dotcircle}	& {\it e}           &
Ta     & {\knrk q}                 		& {\it \d{t}a}      &
sha     & {\knrk z}                 		& {\it \'sa}        \cr
ai.      & {\knrk E} or {\knrk<\dotcircle>}  & {\it ai}          &
Tha    & {\knrk Q}                		& {\it \d{t}ha}     &
Sa     & {\knrk S}                 		& {\it \d{s}a}      \cr
o       & {\knrk o} or {\knrk<\dotcircle A} & {\it o}           &
Da     & {\knrk w}                 		& {\it \d{d}a}      &
sa      & {\knrk s}                 		& {\it sa}          \cr
au.      & {\knrk O} or {\knrk<\dotcircle*}  & {\it au}	        &
Dha    & {\knrk W}                		& {\it \d{d}ha}     &
ha      & {\knrk h}                  		& {\it ha}          \cr
n.	    & {\knrk\dotcircle/}				& {\it \.m}			&
Na     & {\knrk N}                 		& {\it \d{n}a}      &
La     & {\knrk L}                  		& {\it \d{l}a}      \cr
m.      & {\knrk\dotcircle M}          	& {\it \d{m}}       &
ta  	& {\knrk t}                  		& {\it ta}		    &
D.a     & {\knrk w\odnukta}               & {\it \d{r}a}	    \cr
H      & {\knrk\dotcircle H}           	& {\it \d{h}}       &
tha     & {\knrk T}                 		& {\it tha}         &
Dh.a    & {\knrk W\odnukta}               & {\it \d{r}ha}     \cr
	    & 								& 					&
da      & {\knrk d}                  		& {\it da}          &
k		& {\knrk{}k\odhalant} 				&					\cr
	    & 								& 					&
% \~{}e   & {\knrk e\knrksecya>}			 	& {\it \^e}        	&
dha     & {\knrk D}                 		& {\it dha}         &
		&								&					\cr
	    & 								& 					&
% \~{}o 	& {\knrk o\knrksecya>}				& {\it \^o}			&
na		& {\knrk n}                  		& {\it na}		    &
        &                       		&               	\cr
\noalign{\hrule height.8pt}
}\egroup 
\vskip 5mm
For dead consonants, the ASCII transcription is the one for the corresponding consonant without the trailing {\tt a} (for example, {\knrk\ K\odhalant} $\rightarrow$ {\tt kh}). The transliteration for consonant-vowel combinations or conjunctive consonants is achieved by treating the leading consonants in the combination as dead consonants ({\it consonant} + {\knrk \dotcircle \odhalant}).  Below we consider one example each for a consonant-vowel combination and a conjunctive consonant.
\begin{itemize}
\item[{\knrk\ 1}.]{\knrk\ G[} = {\knrk\ G\odhalant} + {\knrk\ i}. Thus, {\knrk\ G[} is transliterated as {\tt ghi},
\item[{\knrk\ 2}.] {\knrk\ \odkSa \odsecma} = {\knrk\ k\odhalant} + {\knrk\ S\odhalant} + {\knrk\ m}. Thus, {\knrk\ \odkSa \odsecma} is transliterated as {\tt kSma}.
\end{itemize}
\begin{warning}
Note that the transliteration scheme is case-sensitive.\\
{\knrk\ k} $\rightarrow$ {\tt ka}, {\knrk\ kA} $\rightarrow$ {\tt kA}, {\knrk\ k[} $\rightarrow$ {\tt ki}, {\knrk\ kX} $\rightarrow$ {\tt kI}, 
{\knrk\ k]} $\rightarrow$ {\tt ku}, {\knrk\ kZ} $\rightarrow$ {\tt kU}.\\
{\knrk\ k\odsignri} $\rightarrow$ {\tt kRu}, {\knrk\ <k} $\rightarrow$ {\tt ke} , {\knrk\ <k>} $\rightarrow$ {\tt kai.}, {\knrk\ <kA} $\rightarrow$ {\tt ko},
{\knrk\ <k*} $\rightarrow$ {\tt kau.}.\\ \\
{\knrk\ ym]nA} $\rightarrow$ {\tt JamunA}, {\knrk\ jh\odsecnalow} $\rightarrow$ {\tt jahna}.
\end{warning}

\subsection{\tt od2tex}
\label{subsection: od2tex}
The pre-processor {\tt od2tex} accepts text files with {\tt .od} extension, containing text in which Odia is typed
in the transliteration scheme described in \ref{subsection: transliterate}. The ideal method for large blocks of Odia text
is to include them in a separate {\tt [TEXNAME].od} file and convert it to {\tt [TEXNAME].tex} using {\tt od2tex}.
The output file {\tt [TEXNAME].tex} contains \odtex\ syntax which tells \TeX\ how to typeset your Odia text. This should be included
via {\tt $\backslash$input\{[TEXNAME].tex\}} in the main \TeX\ file at the appropriate place. For small chunks of text, the conversion 
may be directly done at the commandline by using the {\tt --conv} option, and the results copied to the main \TeX\ file.

We quote the help documentation for {\tt od2tex} verbatim below.
\begin{verbatim}
Usage: od2tex [OPTION]
   or: od2tex [ODNAME[.od]]
   or: od2tex [ODNAME[.od]] [TEXNAME[.tex]]
  Run od2tex on ODNAME.od which contains transliterated text (in English), 
  creating a TeX file ODNAME.tex with Odia-TeX syntax in the same directory,
  if output file is not specified.

  Else run od2tex on ODNAME.od and output to TEXNAME.tex
  
  
-h, --help : View this help file to learn usage
-c, --conv : Convert transliterated text string to Odia-TeX syntax

EXAMPLE(s) :
$ od2tex --conv "oD.ishA"
This is od2tex, Version 0.5 (Odia-Tex, 2016)

Converting the string 'oD.ishA' to Odia-TeX syntax : 
ow\odnukta [zA

$ od2tex init.od
This is od2tex, Version 0.5 (Odia-Tex, 2016)
Converting init.od to init.tex.

$ od2tex init.od final.tex
This is od2tex, Version 0.5 (Odia-Tex, 2016)
Converting init.od to final.tex. 



Written by :
  Soumyashant Nayak
  github page: https://github.com/nsoum/odia-tex/src

Email bug-reports to odia.bhashakosha@gmail.com.
\end{verbatim}

\begin{warning}
Although {\tt od2tex} will process any text file, it is good practice to name the files containing the transliterated text, with the {\tt .od} extension for easier
organization. {\tt od2tex}  will display a warning if the file does not have the extension {\tt .od} though and append {\tt .tex} to the filename after processing.
\end{warning}

\subsection{Example}
Create a file {\tt example.od} which contains transliterated text:
\begin{verbatim}
kaTaka jillAra hariharapura praganA madhyare goTie grAma,
nAma pATapura | grAma muNDAmuNDire goTie ghara | Agili 
pichhili chAri bakharA, khanjA pAchiri chALiAre DhinkishALa,
agaNA madhyare kUa, Agaku dANDaduAra, pachhaku bAD.iduAra |
\end{verbatim}
Run {\tt od2tex example.od} to obtain the output file {\tt example.tex} which contains the text in \odtex\ syntax :
\begin{verbatim}
kqk j[\odlla Ar hr[hrp]r p\odsecra gnA m\oddhya <r <gAq[e g\odsecra Am,
nAm pAqp]r . g\odsecra Am m]\odNDa Am]\odNDa [<r <gAq[e Gr . aAg[l[ 
p[C[l[ cAr[ bKrA, K\odnyja A pAc[r[ cAL[aA<r W[\odngka [zAL,
agNA m\oddhya <r kZa, aAgk] dA\odNDa d]aAr, pCk] bAw\odnukta [d]aAr .
\end{verbatim}
Then {\tt example.tex} is invoked from the main \TeX\ file using the {\tt $\backslash$input} command. A sample \LaTeX\ file and its final output are given below.

\subsubsection{Sample \LaTeX\ file}

\begin{verbatim}
\documentclass[12pt]{article}
\usepackage{odia}

\begin{document}
Text in Konark font :\\
{\knrk
\input{example.tex}
}\\ \\
Text in Cuttack font :\\
{\cttck
\input{example.tex}
}
\end{document}
\end{verbatim}

\subsubsection{Final output}
Text in Konark font :\\
{\knrk{}
kqk j[\odlla Ar hr[hrp]r p\odsecra gnA m\oddhya <r <gAq[e g\odsecra Am,
nAm pAqp]r . g\odsecra Am m]\odNDa Am]\odNDa [<r <gAq[e Gr . aAg[l[ 
p[C[l[ cAr[ bKrA, K\odnyja A pAc[r[ cAL[aA<r W[\odngka [zAL,
agNA m\oddhya <r kZa, aAgk] dA\odNDa d]aAr, pCk] bAw\odnukta [d]aAr .
}\\ \\
Text in Cuttack font :\\
{\cttck{}
kqk j[\odlla Ar hr[hrp]r p\odsecra gnA m\oddhya <r <gAq[e g\odsecra Am,
nAm pAqp]r . g\odsecra Am m]\odNDa Am]\odNDa [<r <gAq[e Gr . aAg[l[ 
p[C[l[ cAr[ bKrA, K\odnyja A pAc[r[ cAL[aA<r W[\odngka [zAL,
agNA m\oddhya <r kZa, aAgk] dA\odNDa d]aAr, pCk] bAw\odnukta [d]aAr .
}

\section{Table of Odia letters}
All basic characters of the Odia script are given, together with their Roman
transliteration, ISCII\footnote{Indian Script Code for Information Interchange, 
IS 13194:1991} and Unicode\footnote{The Unicode Standard, Version 1.0, Vol 1., 
and checked against the tables of version 2.014} code-position, ASCII representation, 
and name. In a few cases, two ISCII characters are needed to represent a single 
Odia character. The vowel signs are given on a dotted circle, which represents 
the consonant or conjunct to which the vowel sign is to be applied.

Conjuncts and ligatures of consonants and vowel signs are given in the
next sections.

\medskip

\halign{\knrk #\quad\hfill&#\quad\hfill&\tt#\ \hfill&\tt#\ \hfill&\tt#\ \hfill&#\hfill\cr
\bf Odia	&\bf Roman	&\bf ISCII	&\bf Unicode	&\bf ASCII	&\bf name	\cr\noalign{\medskip\leftline{\it vowels}\medskip}
%
a		&a		&A4		&0B05		&a	&vowel a 	\cr
aA		&\=a		&A5		&0B06		&A	&vowel aa 	\cr
i		&i		&A6		&0B07		&i	&vowel i	\cr
I		&\=\i		&A7		&0B08		&I	&vowel ii	\cr
u		&u		&A8		&0B09		&u	&vowel u	\cr
U		&\=u		&A9		&0B0A		&U	&vowel uu	\cr
\odvowelri	&\d{r}		&AA		&0B0B		&Ru	&vowel ri	\cr
\odvowelrii	&\d{\=r}		&AA E9		&0B60		&RU	&vowel rii	\cr
\odvowelli	&\d{l}		&A6 E9		&0B0C		&L.u	&vowel li	\cr
\odvowellii	&\d{\=l}		&A7 E9 		&0B61		&L.U	&vowel lii	\cr
e		&e		&AC		&0B0F		&e	&vowel e	\cr
E		&ai		&AD		&0B10		&ai.	&vowel ai	\cr
o		&o		&B0		&0B13		&o	&vowel o	\cr
O		&au		&B1		&0B14		&au.	&vowel au	\cr\noalign{\medskip\leftline{\it vowel modifiers}\medskip}
%
a/		&a\.m		&A1		&0B01		&n.	&candrabindu	\cr
aM		&a\d{m}		&A2		&0B02		&m.	&anusvar	\cr
aH		&a\d{h}		&A3		&0B03		&H	&visarg		\cr\noalign{\medskip\leftline{\it consonants}\medskip}
%
k		&ka		&B3		&0B15		&ka	&consonant ka	\cr
K		&kha		&B4		&0B16		&kha	&consonant kha	\cr
g		&ga		&B5		&0B17		&ga	&consonant ga	\cr
G		&gha		&B6		&0B18		&gha	&consonant gha	\cr
f		&\.na		&B7		&0B19		&nga.	&consonant nga	\cr\noalign{\medskip}
%
c		&ca		&B8		&0B1A		&cha	&consonant ca	\cr
C		&cha		&B9		&0B1B		&chha	&consonant cha	\cr
j		&ja		&BA		&0B1C		&ja	&consonant ja	\cr
J		&jha		&BB		&0B1D		&jha	&consonant jha	\cr
F		&\~na		&BC		&0B1E		&nya.	&consonant nya	\cr\noalign{\medskip}
%
q		&\d{t}a		&BD		&0B1F		&Ta	&consonant tta	\cr
Q		&\d{t}ha	&BE		&0B20		&Tha	&consonant ttha	\cr
w		&\d{d}a		&BF		&0B21		&Da	&consonant dda	\cr
W		&\d{d}ha	&C0		&0B22		&Dha	&consonant ddha	\cr
N		&\d{n}a		&C1		&0B23		&Na	&consonant nna	\cr\noalign{\medskip}
%
t		&ta		&C2		&0B24		&ta	&consonant ta	\cr
T		&tha		&C3		&0B25		&tha	&consonant tha	\cr
d		&da		&C4		&0B26		&da	&consonant da	\cr
D		&dha		&C5		&0B27		&dha	&consonant dha	\cr
n		&na		&C6		&0B28		&na	&consonant na	\cr\noalign{\medskip}
%
p		&pa		&C8		&0B2A		&pa	&consonant pa	\cr
P		&pha		&C9		&0B2B		&pha	&consonant pha	\cr
b		&ba		&CA		&0B2C		&ba	&consonant ba	\cr
B		&bha		&CB		&0B2D		&bha	&consonant bha	\cr
m		&ma		&CC		&0B2E		&ma	&consonant ma	\cr\noalign{\medskip}
%
y		&ya		&CD		&0B5F		&Ja	&consonant ya	\cr
Y		&\.ya		&CE		&0B2F		&ya	&consonant yya	\cr
r		&ra		&CF		&0B30		&ra	&consonant ra	\cr
l		&la		&D1		&0B32		&la	&consonant la	\cr
v		&va		&D4		&0B13 0B4D 0B2C		&va	&consonant va	\cr
\odbadot&ba		&--		&--		&b.		&consonant ba with dot \cr
z		&\'sa		&D5		&0B36		&sha	&consonant sha	\cr
S		&\d{s}a		&D6		&0B37		&Sa	&consonant ssa	\cr
s		&sa		&D7		&0B38		&sa	&consonant sa	\cr
h		&ha		&D8		&0B39		&ha	&consonant ha	\cr
L		&\d{l}a		&D2		&0B33		&La	&consonant lla	\cr\noalign{\medskip}
%
w\odnukta	&\d{r}a		&BF E9		&0B5C		&D.a	&consonant rra	\cr
W\odnukta	&\d{r}ha	&C0 E9		&0B5D		&Dh.a	&consonant rrha	\cr\noalign{\medskip\leftline{\it vowel signs}\medskip}
%
\dotcircle A	&\=a		&DA		&0B3E		&A	&vowel sign aa	\cr
\dotcircle[	&i		&DB		&0B3F		&i	&vowel sign i	\cr
\dotcircle X	&\=\i		&DC		&0B40		&I	&vowel sign ii	\cr
\dotcircle]	&u		&DD		&0B41		&u	&vowel sign u	\cr
\dotcircle Z	&\=u		&DE		&0B42		&U	&vowel sign uu	\cr
\dotcircle\odsignri&\d{r}		&DF		&0B43		&Ru	&vowel sign ri	\cr
\dotcircle\odsignrii&\d{\=r}		&DF E9		&--		&RU	&vowel sign rii	\cr
\dotcircle\odsignli&\d{l}		&DB E9		&--		&L.u	&vowel sign li	\cr
\dotcircle\odsignlii&\d{\=l}		&DC E9		&--		&L.U	&vowel sign lii	\cr
<\dotcircle	&e		&E1		&0B47		&e	&vowel sign e	\cr
<\dotcircle>	&ai		&E2		&0B48		&ai.	&vowel sign ai	\cr
<\dotcircle A	&o		&E4		&0B4B		&o	&vowel sign o	\cr
<\dotcircle*	&au		&E5		&0B4C		&au.	&vowel sign au	\cr\noalign{\medskip\leftline{\it additional vowels}\medskip}
%
e\odsecya>	&\^e	&--	&0BOF 0B4D 0B5F	&ai.	\cr
o\odsecya>	&\^o	&--	&0B13 0B4D 0B5F	&au.	\cr\noalign{\medskip\leftline{\it other signs and symbols}\medskip}
%
\dotcircle+	&		&E8		&0B4D		&$\backslash$odhalant	&halant		\cr
\odavagraha	&		&EA E9		&0B3D		&$\backslash$odavagraha 	&avagraha	\cr
\odganesh	&		&--		&0B70		&$\backslash$odganesh	&isshar		\cr
\odomsign	&		&A1 E9		&--		&$\backslash$odomsign	&om sign	\cr
\odpercent & & &0025 &$\backslash$odpercent &percent sign   \cr
\odrupee & &  &20B9 &$\backslash$odrupee &Indian rupee symbol \cr\noalign{\medskip\leftline{\it digits}\medskip}
%
0		&0		&F1		&0B66		&0	&digit zero	\cr
1		&1		&F2		&0b67		&1	&digit one	\cr
2		&2		&F3		&0B68		&2	&digit two	\cr
3		&3		&F4		&0B69		&3	&digit three	\cr
4		&4		&F5		&0B6A		&4	&digit four	\cr
5		&5		&F6		&0B6B		&5	&digit five	\cr
6		&6		&F7		&0B6C		&6	&digit six	\cr
7		&7		&F8		&0B6D		&7	&digit seven	\cr
8		&8		&F9		&0B6E		&8	&digit eight	\cr
9		&9		&FA		&0B6F		&9	&digit nine	\cr\noalign{\medskip\leftline{\it conjunct control}\medskip}
%
\odkra	&kra	&B3 E8 CF		&0B15 0B4D 0B30	&kra	&{\it ordinary conjunct}\cr
k+r		&kra	&B3 E8 E8 CF	&0B15 0B4D 200C 0B30	&k\{\}ra	&{\it explicit halant}\cr
k\odsecra	&kra	&B3 E8 E8 E8 CF	&0B15 0B4D 200D 0B30	&k$\backslash$odsecra 	&{\it alternate conjunct}\cr
}

\section{Table of Consonant-Vowel Sign Combinations}

Vowel-signs often combine with the consonant or conjunct they modify.\\
\halign{\it#\quad\hfill&&\knrk#\quad\hfill\cr
	&\it a	&\it\=a &\it i	&\it\=\i &\it u	&\it\=u &\it\d{r} &\it e &\it ai &\it o &\it au &\it \.m&\it \d{m}&\it \d{h}\cr
	&a	&aA	&i	&I	&u	&U	&\odvowelri	 &e	&E	&o	&O	&a/	&aM	&aH	\cr\noalign{\smallskip}
k	&k	&kA	&k[	&kX	&k]	&kZ	&k\odsignri &<k	&<k>	&<kA	&<k*	&k/	&kM	&kH	\cr
kh	&K	&KA	&K[	&KX	&K]	&KZ	&K\odsignri &<K	&<K>	&<KA	&<K*	&K/	&KM	&KH	\cr
g	&g	&gA	&g[	&gX	&g]	&gZ	&g\odsignri &<g	&<g>	&<gA	&<g*	&g/	&gM	&gH	\cr
gh	&G	&GA	&G[	&GX	&G]	&GZ	&G\odsignri &<G	&<G>	&<GA	&<G*	&G/	&GM	&GH	\cr
\.n	&f	&fA	&f[	&fX	&f]	&fZ	&f\odsignri &<f	&<f>	&<fA	&<f*	&f/	&fM	&fH	\cr\noalign{\smallskip}
%
c	&c	&cA	&c[	&cX	&c]	&cZ	&c\odsignri &<c	&<c>	&<cA	&<c*	&c/	&cM	&cH	\cr
ch	&C	&CA	&C[	&CX	&C]	&CZ	&C\odsignri &<C	&<C>	&<CA	&<C*	&C/	&CM	&CH	\cr
j	&j	&jA	&j[	&jX	&j]	&jZ	&j\odsignri &<j	&<j>	&<jA	&<j*	&j/	&jM	&jH	\cr
jh	&J	&JA	&J[	&JX	&J]	&JZ	&J\odsignri &<J	&<J>	&<JA	&<J*	&J/	&JM	&JH	\cr
\~n	&F	&FA	&F[	&FX	&F]	&FZ	&F\odsignri &<F	&<F>	&<FA	&<F*	&F/	&FM	&FH	\cr\noalign{\smallskip}
%
\d{t}	&q	&qA	&q[	&qX	&q]	&qZ	&q\odsignri &<q	&<q>	&<qA	&<q*	&q/	&qM	&qH	\cr
\d{t}h	&Q	&QA	&Q[	&QX	&Q]	&QZ	&Q\odsignri &<Q	&<Q>	&<QA	&<Q*	&Q/	&QM	&QH	\cr
\d{d}	&w	&wA	&w[	&wX	&w]	&wZ	&w\odsignri &<w	&<w>	&<wA	&<w*	&w/	&wM	&wH	\cr
\d{d}h	&W	&WA	&W[	&WX	&W]	&WZ	&W\odsignri &<W	&<W>	&<WA	&<W*	&W/	&WM	&WH	\cr
\d{n}	&N	&NA	&N[	&NX	&N]	&NZ	&N\odsignri &<N	&<N>	&<NA	&<N*	&N/	&NM	&NH	\cr\noalign{\smallskip}
%
t	&t	&tA	&t[	&tX	&t]	&tZ	&t\odsignri &<t	&<t>	&<tA	&<t*	&t/	&tM	&tH	\cr
th	&T	&TA	&T[	&TX	&T]	&TZ	&T\odsignri &<T	&<T>	&<TA	&<T*	&T/	&TM	&TH	\cr
d	&d	&dA	&d[	&dX	&d]	&dZ	&d\odsignri &<d	&<d>	&<dA	&<d*	&d/	&dM	&dH	\cr
dh	&D	&DA	&D[	&DX	&D]	&DZ	&D\odsignri &<D	&<D>	&<DA	&<D*	&D/	&DM	&DH	\cr
n	&n	&nA	&n[	&nX	&n]	&nZ	&n\odsignri &<n	&<n>	&<nA	&<n*	&n/	&nM	&nH	\cr\noalign{\smallskip}
%
p	&p	&pA	&p[	&pX	&p]	&pZ	&p\odsignri &<p	&<p>	&<pA	&<p*	&p/	&pM	&pH	\cr
ph	&P	&PA	&P[	&PX	&P]	&PZ	&P\odsignri &<P	&<P>	&<PA	&<P*	&P/	&PM	&PH	\cr
b	&b	&bA	&b[	&bX	&b]	&bZ	&b\odsignri &<b	&<b>	&<bA	&<b*	&b/	&bM	&bH	\cr
bh	&B	&BA	&B[	&BX	&B]	&BZ	&B\odsignri &<B	&<B>	&<BA	&<B*	&B/	&BM	&BH	\cr
m	&m	&mA	&m[	&mX	&m]	&mZ	&m\odsignri &<m	&<m>	&<mA	&<m*	&m/	&mM	&mH	\cr\noalign{\smallskip}
%
y	&Y	&YA	&Y[	&YX	&Y]	&YZ	&Y\odsignri &<Y	&<Y>	&<YA	&<Y*	&Y/	&YM	&YH	\cr
\.{y}	&y	&yA	&y[	&yX	&y]	&yZ	&y\odsignri &<y	&<y>	&<yA	&<y*	&y/	&yM	&yH	\cr
r	&r	&rA	&r[	&rX	&r]	&rZ	&r\odsignri &<r	&<r>	&<rA	&<r*	&r/	&rM	&rH	\cr
l	&l	&lA	&l[	&lX	&l]	&lZ	&l\odsignri &<l	&<l>	&<lA	&<l*	&l/	&lM	&lH	\cr
%v	&v	&vA	&v[	&vX	&v]	&vZ	&v\odsignri &<v	&<v>	&<vA	&<v*	&v/	&vM	&vH	\cr
\'s	&z	&zA	&z[	&zX	&z]	&zZ	&z\odsignri &<z	&<z>	&<zA	&<z*	&z/	&zM	&zH	\cr
\d{s}	&S	&SA	&S[	&SX	&S]	&SZ	&S\odsignri &<S	&<S>	&<SA	&<S*	&S/	&SM	&SH	\cr
s	&s	&sA	&s[	&sX	&s]	&sZ	&s\odsignri &<s	&<s>	&<sA	&<s*	&s/	&sM	&sH	\cr
h	&h	&hA	&h[	&hX	&h]	&hZ	&h\odsignri &<h	&<h>	&<hA	&<h*	&h/	&hM	&hH	\cr
\d{l}	&L	&LA	&L[	&LX	&L]	&LZ	&L\odsignri &<L	&<L>	&<LA	&<L*	&L/	&LM	&LH	\cr\noalign{\smallskip}
	&\dotcircle 	&\dotcircle A	&\dotcircle [	
&\dotcircle X	&\dotcircle ]	&\dotcircle Z	&\dotcircle\odsignri
&<\dotcircle 	&<\dotcircle >	&<\dotcircle A	&<\dotcircle *
&\dotcircle / &\dotcircle M	&\dotcircle H	\cr
}
\subsection{combinations of vowel signs, reph and candrabindu}
When vowel signs, reph and candrabindu appear together, the following
ligatures are used.
\hspace{0.1in}  {\knrk
\dotcircle[\odcandrabindu\ %
\dotcircle[\odreph\ %
\dotcircle[\odreph\odcandrabindu\ %
<\dotcircle>\odcandrabindu\ %
<\dotcircle>\odreph\ %
<\dotcircle>\odreph\odcandrabindu\ %
<\dotcircle*\odcandrabindu\ %
<\dotcircle*\odreph\ %
<\dotcircle*\odreph\odcandrabindu\ %
}


\section{Table of Conjunct Consonants}
The table below gives all the conjuncts included in the font.

\twocolumn

\halign{%
\knrk#\rm\ \hfill&#\  \hfill&
\knrk#\rm\ \hfill&#\  \hfill&
\knrk#\rm\ \hfill&#\  \hfill&
\knrk#\rm\ \hfill&#\  \hfill&
\knrk#\rm\quad\quad\hfill&#\cr
k\odhalant	&+&k	&&	&&	&=&\odkka		&kka		\cr
k\odhalant   &+&q    &&  &&  &=&\odkTa		&k\d{t}a	\cr
k\odhalant	&+&t	&&	&&	&=&\odkta		&kta		\cr
k\odhalant	&+&r	&&	&&	&=&\odkra {\rm\ or} k\odsecra	&kra	\cr
k\odhalant	&+&l	&&	&&	&=&k\odsecla	&kla		\cr
k\odhalant	&+&b	&&	&&	&=&k\odsecva	&kva		\cr
k\odhalant	&+&S	&&	&&	&=&\odkSa		&k\d{s}a	\cr
k\odhalant	&+&S\odhalant	&+&N	&&	&=&\odkSNa	&k\d{s}\d{n}a	\cr
k\odhalant	&+&s	&&	&&	&=&\odksa		&ksa		\cr
%
K\odhalant	&+&Y	&&	&&	&=&K\odsecya	&khya		\cr
%
g\odhalant	&+&g	&&	&&	&=&\odgga		&gga		\cr
g\odhalant	&+&D	&&	&&	&=&\odgdha		&gdha		\cr
g\odhalant	&+&n	&&	&&	&=&g\odsecna	&gna		\cr
g\odhalant	&+&r	&&	&&	&=&g\odsecra	&gra		\cr
g\odhalant	&+&l	&&	&&	&=&g\odsecla	&gla		\cr
%
G\odhalant	&+&n	&&	&&	&=&G\odsecna	&ghna		\cr
%
f\odhalant	&+&k	&&	&&	&=&\odngka		&\.nka		\cr
f\odhalant	&+&K	&&	&&	&=&\odngkha		&\.nkha		\cr
f\odhalant	&+&g	&&	&&	&=&\odngga		&\.nga		\cr
f\odhalant	&+&G	&&	&&	&=&\odnggha		&\.ngha		\cr
%
c\odhalant	&+&c	&&	&&	&=&\odcca		&cca		\cr
c\odhalant	&+&C	&&	&&	&=&\odccha		&ccha		\cr
%
j\odhalant	&+&j	&&	&&	&=&j\odsecja		&jja		\cr
j\odhalant	&+&J	&&	&&	&=&j\odsecjha		&jjha		\cr
j\odhalant	&+&F	&&	&&	&=&\odjnya		&j\~na		\cr
j\odhalant	&+&Y	&&	&&	&=&j\odsecya	&jya		\cr
j\odhalant	&+&b	&&	&&	&=&j\odsecva	&jva		\cr
%
F\odhalant	&+&c	&&	&&	&=&\odnyca		&\~nca		\cr
F\odhalant	&+&C	&&	&&	&=&\odnyca\odseccha	&\~ncha		\cr
F\odhalant	&+&j	&&	&&	&=&\odnyja		&\~nja		\cr
F\odhalant	&+&J	&&	&&	&=&\odnyjha		&\~njha		\cr
%
q\odhalant	&+&q	&&	&&	&=&\odTTa		&\d{t}\d{t}a	\cr
%
w\odhalant   &+&g	&&	&&	&=&\odDga		&\d{d}ga		\cr
w\odnukta\odhalant&+&g&&	&&	&=&\odRga		&\d{r}ga		\cr
w\odhalant	&+&w	&&	&&	&=&\odDDa		&\d{d}\d{d}a	\cr
%
N\odhalant	&+&q	&&	&&	&=&\odNTa		&\d{n}\d{t}a	\cr
N\odhalant	&+&Q	&&	&&	&=&\odNTha		&\d{n}\d{t}ha	\cr
N\odhalant	&+&w	&&	&&	&=&\odNDa		&\d{n}\d{d}a	\cr		
N\odhalant	&+&W	&&	&&	&=&\odNDha		&\d{n}\d{d}ha	\cr
N\odhalant	&+&N	&&	&&	&=&\odNNa		&\d{n}\d{n}a	\cr
%
t\odhalant	&+&k    &&  	&& 	&=&\odtka		&tka		\cr
t\odhalant	&+&t	&&	&&	&=&\odtta		&tta		\cr
t\odhalant	&+&n	&&	&&	&=&\odtna		&tna		\cr
t\odhalant      &+&p	&&	&&	&=&\odtpa		&tpa		\cr
t\odhalant	&+&m	&&	&&	&=&\odtma {\rm\ or} t\odsecma	&tma	\cr
t\odhalant	&+&r	&&	&&	&=&\odtra {\rm\ or} t\odsecra	&tra	\cr
t\odhalant	&+&s	&&	&&	&=&\odtsa		&tsa		\cr
%
d\odhalant	&+&g	&&	&&	&=&\oddga		&dga		\cr
d\odhalant	&+&d	&&	&&	&=&\oddda		&dda		\cr
d\odhalant	&+&D	&&	&&	&=&\odddha		&ddha		\cr
d\odhalant	&+&b	&&	&&	&=&\oddba		&dba		\cr
d\odhalant	&+&B	&&	&&	&=&\oddbha		&dbha		\cr
d\odhalant	&+&m	&&	&&	&=&d\odsecma	&dma		\cr
%
D\odhalant	&+&Y	&&	&&	&=&\oddhya		&dhya		\cr
D\odhalant	&+&b	&&	&&	&=&D\odsecva	&dhva		\cr
%
n\odhalant	&+&t	&&	&&	&=&\odnta		&nta		\cr
n\odhalant	&+&t	&+&Y	&&	&=&\odnta\odsecya	&ntya		\cr
n\odhalant	&+&t	&+&r	&&	&=&\odntra	&ntra		\cr
n\odhalant	&+&T	&&	&&	&=&n\odsectha	&ntha		\cr
n\odhalant	&+&d	&&	&&	&=&\odnda		&nda		\cr
n\odhalant	&+&D	&&	&&	&=&\odndha		&ndha		\cr
n\odhalant	&+&n	&&	&&	&=&n\odsecna	&nna		\cr
%
p\odhalant	&+&Y	&&	&&	&=&p\odsecya	&pya		\cr
%
b\odhalant	&+&j	&&	&&	&=&b\odsecja		&bja		\cr
b\odhalant	&+&d	&&	&&	&=&\odbda		&bda		\cr
b\odhalant   &+&D	&&	&&	&=&\odbdha		&bdha		\cr
b\odhalant	&+&b	&&	&&	&=&\odbba		&bba		\cr
%
p\odhalant	&+&t	&&	&&	&=&\odpta		&pta		\cr
p\odhalant	&+&s	&&	&&	&=&\odpsa		&psa		\cr
%
m\odhalant	&+&p	&&	&&	&=&\odmpa		&mpa		\cr
m\odhalant	&+&P	&&	&&	&=&\odmpha		&mpha		\cr
m\odhalant	&+&b	&&	&&	&=&\odmba	&mba		\cr
m\odhalant	&+&B	&&	&&	&=&\odmbha	&mbha		\cr
m\odhalant	&+&m	&&	&&	&=&\odmma		&mma		\cr
%
r\odhalant	&+&N	&+&N	&&	&=&\odNNa\odreph	&r\d{n}\d{n}a	\cr
r\odhalant	&+&b	&&	&&	&=&b\odreph		&rba		\cr
%
L\odhalant	&+&k	&&	&&	&=&\odLka		&lka		\cr
L\odhalant	&+&p	&&	&&	&=&\odLpa		&lpa		\cr
L\odhalant	&+&P	&&	&&	&=&\odLpha		&lpha		\cr
l\odhalantlow	&+&l	&&	&&	&=&\odlla		&lla		\cr
%
z\odhalant	&+&c	&&	&&	&=&\odshca		&\'sca		\cr
z\odhalant	&+&C	&&	&&	&=&z\odseccha	&\'scha		\cr
z\odhalant	&+&q	&&	&&	&=&\odshTa		&\'s\d{t}a	\cr
z\odhalant	&+&n	&&	&&	&=&z\odsecna	&\'sna		\cr
%
S\odhalant	&+&k	&&	&&	&=&\odSka		&\d{s}ka	\cr
S\odhalant	&+&q	&&	&&	&=&\odSTa		&\d{s}\d{t}a	\cr
S\odhalant	&+&Q	&&	&&	&=&\odSTha		&\d{s}\d{t}ha	\cr
S\odhalant   &+&N    &&  &&  &=&\odSNa		&\d{s}\d{n}a	\cr
S\odhalant	&+&p	&&	&&	&=&\odSpa		&\d{s}pa	\cr
S\odhalant	&+&P	&&	&&	&=&\odSpha		&\d{s}pha	\cr
%
s\odhalant	&+&k	&&	&&	&=&\odska		&ska		\cr
s\odhalant	&+&K	&&	&&	&=&\odskha		&skha		\cr
s\odhalant	&+&t	&&	&&	&=&\odsta		&sta		\cr
s\odhalant	&+&t	&+&r	&&	&=&\odstra	&stra		\cr
s\odhalant	&+&T	&&	&&	&=&s\odsectha	&stha		\cr
s\odhalant	&+&p	&&	&&	&=&\odspa		&spa		\cr
s\odhalant	&+&P	&&	&&	&=&\odspha		&spha		\cr
%
h\odhalantlow	&+&n	&&	&&	&=&h\odsecna		&hna		\cr
h\odhalantlow	&+&m	&&	&&	&=&h\odsecma		&hma		\cr
h\odhalantlow	&+&b	&&	&&	&=&\odhva		&hva		\cr
}


%\section{Glyph Chart}

%\section{Concluding Remarks}


\end{document}